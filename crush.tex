%% ``Let us change our traditional attitude to the construction of% ===> this file was generated automatically by noweave --- better not edit it
%% programs: Instead of imagining that our main task is to instruct a
%% computer what to do, let us concentrate rather on explaining to
%% humans what we want the computer to do.'' -- Donald E. Knuth, 1984.

% 1  one inch + \hoffset      2  one inch + \voffset
% 3  \oddsidemargin = 22pt    4  \topmargin = 22pt
%   or \evensidemargin
% 5  \headheight = 13pt       6  \headsep = 19pt
% 7  \textheight = 595pt      8  \textwidth = 360pt
% 9  \marginparsep = 7pt     10  \marginparwidth = 106pt
% 11 \footskip = 27pt            \marginparpush = 5pt (not shown)
%    \hoffset = 0pt              \voffset = 0pt
%    \paperwidth = 597pt         \paperheight = 845pt

%\setlength{\oddsidemargin}{0.5cm}
%\setlength{\evensidemargin}{0.5cm}
%\setlength{\marginparsep}{0.75cm}
%\setlength{\marginparwidth}{2.5cm}
%\setlength{\marginparpush}{1.0cm}
%\setlength{\textwidth}{150mm}

\documentclass[a4paper,11pt]{article}
\usepackage[text={6.75in,10in},centering]{geometry}

%% \usepackage{geometry}
%%  \geometry{
%%  a4paper,
%%  total={170mm,257mm},
%%  left=20mm,
%%  top=20mm,
%%  }

\usepackage[backend=biber]{biblatex}
\addbibresource{refs.bib}
%% \renewcommand{\cite}{\parencite}

\usepackage[T1]{fontenc}
\usepackage[utf8]{inputenc}
%\usepackage[portuguese]{babel}
\usepackage{hyperref}
\usepackage{amsmath,amsthm,amssymb}
\allowdisplaybreaks
\usepackage{lmodern}
\usepackage{noweb}
\noweboptions{longchunks,longxref,smallcode}
% redefine this command so we can allow code to break across
% multiple pages

%% \def\nwendcode{\endtrivlist \endgroup \vfil\penalty10\vfilneg}
%% \let\nwdocspar=\smallbreak

\def\nwendcode{\endtrivlist \endgroup}
\let\nwdocspar=\par

\author{Daniel Chicayban Bastos\\
        Luis Antonio Brasil Kowada}

\title{The implementation of {\Tt{}\Rm{}{\it{}crush}\nwendquote}}
\date{May 5th 2019}

\begin{document}
\fontfamily{cmr}\selectfont
\maketitle

%\setlength{\parskip}{7pt}
%\setlength{\parindent}{0pt}

\section*{This document}

This document describes the implementation of {\Tt{}\Rm{}{\it{}crush}\nwendquote}, a program for
testing random number generators using the TestU01
library~\cite{l2005testu01,l2007testu01,l2009testu01}. It is at the
same time the program's source code and its documentation: it is
written in the literate programming~\cite{knuth1984literate,
knuth1994cweb, ramsey1994literate} spirit.  

A literate program interleaves program source code and documentation
in the same document.  When chunks of code are displayed on a page of
this document, you will find an integer on the left margin of that
page.  This integer specifies the page on which the chunk of code is
written.

% Flushed to the right, you will find a number written between
% parentheses.  That number references the page on which the code chunk
% is used.  So, when you're reading a chunk, you can look to the right
% and you will know which page that chunk is used.  If no such number
% appear, it's because a root chunk is being defined.  A root chunk is
% used when we define a new source code file.

If a chunk is referenced by an integer and by a letter, that means
there's more than one chunk defined on the same page, so letters
distinguish chunks defined on the same page.

\section*{The {\Tt{}\Rm{}{\it{}crush}\nwendquote} program}

The program implemented in this document is called {\Tt{}\Rm{}{\it{}crush}\nwendquote}.

\begin{verbatim}
%./crush --help
Usage: ./crush [options]
 Tests your data for randomness against TestU01.

Examples:
cat /dev/urandom | crush -b small -n 'the local generator'
xorshift32 | crush -b small -n xorshift32

 The options are:
-b, --battery    Your choice of battery (small, medium, big)
-n, --name       The name of your generator
-h, --help       Display this information
%
\end{verbatim}

Suppose we have a program called {\Tt{}\Rm{}{\it{}xs32}\nwendquote}, which implements George
Marsaglia's {\em xorshift} pseudo-random number generator with an
output size of 32 bits.  Assume {\Tt{}\Rm{}{\it{}xs32}\nwendquote} writes its random numbers in
binary, in little-endian format, to the standard output.  To test this
generator against the small battery of the library TestU01, we can say
the following to the UNIX shell:

\begin{verbatim}
%./xs32 | crush -b small -n xs32
[...]
%
\end{verbatim}

Where we wrote ``[...]'', we would see a report from the library
describing the results of each statistical test applied to the
generator.  

The option {\Tt{}\Rm{}-{\it{}n}\nwendquote} sets the name of the generator, a mere formality of
the TestU01 library.  As the library produces the report, it
conveniently annotates it with the generator's name for our later
reference.

It the next sections of this document, we write the code that
implements this program and we take the opportunity to explain our
strategies.

\subsection*{Typical usage of the library TestU01}

Typical usage of the library TestU01 involves writing the RNG as a C
function and then passing a pointer to this function the TestU01's
library interfaces for testing the generator.  The purpose of
{\Tt{}\Rm{}{\it{}crush}\nwendquote} is to allow any generator written in any language to take
advantage of the facitilities of the library TestU01, without
requiring the user interested in testing the generator to write the
generator in the C programming language.

\subsection*{The chunks of {\Tt{}\Rm{}{\it{}crush}\nwendquote}}

The program {\Tt{}\Rm{}{\it{}crush}\nwendquote} is composed of the following chunks.  We
present them first so you can immediately see its familiar strucutre
of a typical C program.  However, we do not describe the chunks in the
order below because most of these chunks are irrelevant to
understanding how the program works.

\def\LA{\begingroup\maybehbox\bgroup\setupmodname\Rm$\langle$}\def\RA{$\rangle$\egroup\endgroup}\providecommand{\MM}{\kern.5pt\raisebox{.4ex}{\begin{math}\scriptscriptstyle-\kern-1pt-\end{math}}\kern.5pt}\providecommand{\PP}{\kern.5pt\raisebox{.4ex}{\begin{math}\scriptscriptstyle+\kern-1pt+\end{math}}\kern.5pt}\def\commopen{/\begin{math}\ast\,\end{math}}\def\commclose{\,\begin{math}\ast\end{math}\kern-.5pt/}\def\begcomm{\begingroup\maybehbox\bgroup\setupmodname}\def\endcomm{\egroup\endgroup}\nwfilename{crush.nw}\nwbegincode{1}\sublabel{NW1iLVcl-1oTVZB-1}\nwmargintag{{\nwtagstyle{}\subpageref{NW1iLVcl-1oTVZB-1}}}\moddef{crush.c~{\nwtagstyle{}\subpageref{NW1iLVcl-1oTVZB-1}}}\endmoddef\Rm{}\nwstartdeflinemarkup\nwenddeflinemarkup
\LA{}header and declarations~{\nwtagstyle{}\subpageref{NW1iLVcl-2UDuHk-1}}\RA{}
\LA{}a generic generator~{\nwtagstyle{}\subpageref{NW1iLVcl-PfqK8-1}}\RA{}
\LA{}the main function~{\nwtagstyle{}\subpageref{NW1iLVcl-1Pi9cp-1}}\RA{}
\LA{}other functions~{\nwtagstyle{}\subpageref{NW1iLVcl-3EEkjx-1}}\RA{}
\nwnotused{crush.c}\nwendcode{}\nwbegindocs{2}\subsection*{A generic generator} 

This is the most important part of {\Tt{}\Rm{}{\it{}crush}\nwendquote}.  Given the typical
usage of TestU01, we would like to implement a generator in C which
gets its source of randomness from the {\Tt{}\Rm{}{\bf{}stdin}\nwendquote} and not from an
arithmetical procedure.  By getting the random numbers from the
{\Tt{}\Rm{}{\bf{}stdin}\nwendquote}, we can feed the program different types of RNG output
without having to write another C program for testing against TestU01.

This generator consumes its bytes in binary, so if you're writing a
RNG, it should write its random numbers to the {\Tt{}\Rm{}{\bf{}stdout}\nwendquote} with, for
example, the {\Tt{}\Rm{}{\it{}fwrite}\nwendquote} function from the standard C library.  We also
assume all data is always written in little-endian.

The bytes will be read into a {\Tt{}\Rm{}{\it{}buffer}\nwendquote} capable of holding 8192
bytes.  Since the data we store in this buffer is of type
{\Tt{}\Rm{}{\bf{}unsigned} {\bf{}int}\nwendquote}, this means that the number of {\Tt{}\Rm{}{\bf{}unsigned} {\bf{}int}\nwendquote} that
can fit into this buffer is {\Tt{}\Rm{}8192\begin{math}\div\end{math}({\bf{}sizeof} ({\bf{}unsigned} {\bf{}int}))\nwendquote}.  We
choose 8192 because it is a usual block size for block devices, so,
when issuing {\Tt{}\Rm{}{\it{}fread}\nwendquote} calls, we hope to carry much as information we
can without using more memory than we must.

Performance is the reason we don't read a single {\Tt{}\Rm{}{\bf{}unsigned} {\bf{}int}\nwendquote} and
return it immediately from the function.  That would be too slow.  So,
the reading logic is as follows.  For {\Tt{}\Rm{}{\it{}buffer}\nwendquote} management, we use
two variables, {\Tt{}\Rm{}{\it{}pos}\nwendquote} and {\Tt{}\Rm{}{\it{}limit}\nwendquote}.  While {\Tt{}\Rm{}{\it{}pos}\nwendquote} is used to mark
the next number that the function must return, {\Tt{}\Rm{}{\it{}limit}\nwendquote} divides the
buffer into two parts, left and right.  To the left of {\Tt{}\Rm{}{\it{}limit}\nwendquote},
there are random numbers read from the {\Tt{}\Rm{}{\bf{}stdin}\nwendquote}; to the right, there
is empty space, space we didn't use yet: we don't know if the
{\Tt{}\Rm{}{\it{}fread}\nwendquote} call will fill up {\Tt{}\Rm{}{\it{}buffer}\nwendquote} completely, so, in such cases,
we must know we've read our last valid number and refill {\Tt{}\Rm{}{\it{}buffer}\nwendquote}
with another {\Tt{}\Rm{}{\it{}fread}\nwendquote} call.

If {\Tt{}\Rm{}{\it{}pos}\nwendquote} would point to a number that's ``on the right side'' of
{\Tt{}\Rm{}{\it{}buffer}\nwendquote}, than we would produce invalid data.  Instead, we must
refill the {\Tt{}\Rm{}{\it{}buffer}\nwendquote}.

If we ran out of data from the {\Tt{}\Rm{}{\bf{}stdin}\nwendquote}, then {\Tt{}\Rm{}{\it{}fread}\nwendquote} returns 0, in
which case we have an error situation.  Either the generator has
stopped producing data, or some other unexpected situation occurred.
We must stop.  A generator should be infinite, so, having nothing
sensible to do, we stop\footnote{This error condition will happen if
you're testing randomness from an RNG whose data is in a file on disk
and there is not enough data in the file for the chosen battery of
tests.  The sensible solution is to get a larger file.  {\tt
BigCrush}, the large battery of tests from the TestU01 library, can
consume up to 1428420566888 bytes of data, which is approximately 1330
GiB.  If you have very little data, the package {\em
PractRand}~\cite{practrand} is more convenient than the library
TestU01 because it incrementally tests your data, using more and more
data from your file up until your generator fails a test, when it
stops, going up to a maximum of 32 TiB of data.}.

Having filled {\Tt{}\Rm{}{\it{}buffer}\nwendquote} as much as possible, we can return a random
number and increment {\Tt{}\Rm{}{\it{}pos}\nwendquote}.  That completes our generic generator.
Now that you have read the strategy, you can see it implemented below
and read again our description above if necessary.
 
\nwenddocs{}\nwbegincode{3}\sublabel{NW1iLVcl-PfqK8-1}\nwmargintag{{\nwtagstyle{}\subpageref{NW1iLVcl-PfqK8-1}}}\moddef{a generic generator~{\nwtagstyle{}\subpageref{NW1iLVcl-PfqK8-1}}}\endmoddef\Rm{}\nwstartdeflinemarkup\nwusesondefline{\\{NW1iLVcl-1oTVZB-1}}\nwenddeflinemarkup
{\bf{}unsigned} {\bf{}int} {\it{}generator}({\bf{}void})
{\nwlbrace}
  {\bf{}static} {\bf{}uint64\_t} {\it{}nbytes};
  {\bf{}static} {\bf{}unsigned} {\bf{}int} {\it{}buffer}[8192 \begin{math}\div\end{math} {\bf{}sizeof} ({\bf{}unsigned} {\bf{}int})];
  {\bf{}static} {\bf{}unsigned} {\bf{}int} {\it{}pos}; {\commopen}\begcomm{} where is our number? \endcomm{}{\commclose}
  {\bf{}static} {\bf{}unsigned} {\bf{}int} {\it{}limit}; {\commopen}\begcomm{} where does the data in the buffer end? \endcomm{}{\commclose}
  
  {\bf{}if} ({\it{}pos} \begin{math}\geq\end{math} {\it{}limit}) {\nwlbrace}
    {\commopen}\begcomm{} refill the buffer and continue by restarting at 0 \endcomm{}{\commclose}
    {\it{}limit} = {\it{}fread}({\it{}buffer}, {\bf{}sizeof}({\bf{}unsigned} {\bf{}int}), ({\bf{}sizeof} {\it{}buffer})\begin{math}\div\end{math}{\bf{}sizeof}({\bf{}unsigned} {\bf{}int}), {\bf{}stdin});
    {\bf{}if} ({\it{}limit} \begin{math}\equiv\end{math} 0) {\nwlbrace}
      \begcomm{}// We read 0 bytes.  This either means we found EOF or we have \endcomm{}
      \begcomm{}// an error.  A decent generator is infinite, so this should never\endcomm{}
      \begcomm{}// happen.\endcomm{}
      {\bf{}if} ({\it{}ferror}({\bf{}stdin}) \begin{math}\neq\end{math} 0) {\nwlbrace}
        {\it{}perror}({\tt{}"fread"}); {\bf{}exit}(-1);
      {\nwrbrace}
      {\bf{}if} ({\it{}feof}({\bf{}stdin}) \begin{math}\neq\end{math} 0) {\nwlbrace}
        {\it{}printf}({\tt{}"generator produced eof after %ld bytes{\char92}n"}, {\it{}nbytes});
        {\bf{}exit}(0);
      {\nwrbrace}
    {\nwrbrace}

    {\it{}nbytes} += {\it{}limit} \begin{math}\ast\end{math} {\bf{}sizeof} ({\bf{}unsigned} {\bf{}int});
    {\it{}pos} = 0;
  {\nwrbrace}
  
  {\bf{}unsigned} {\bf{}int} {\it{}random} = {\it{}buffer}[{\it{}pos}]; {\commopen}\begcomm{} get one \endcomm{}{\commclose}
  {\it{}pos} += 1;
  {\bf{}return} {\it{}random};
{\nwrbrace}
\nwindexdefn{\nwixident{generator}}{generator}{NW1iLVcl-PfqK8-1}\eatline
\nwused{\\{NW1iLVcl-1oTVZB-1}}\nwidentdefs{\\{{\nwixident{generator}}{generator}}}\nwendcode{}\nwbegindocs{4}%
\subsection*{The {\Tt{}\Rm{}{\it{}main}\nwendquote} function}

This program is essentially the generic function we implemented above.
Everything is straightforward now.  In {\Tt{}\Rm{}{\it{}main}\nwendquote}, we just instantiate a
generator and run it against a particular battery chosen by the user.
The generator is referenced by the structure {\Tt{}\Rm{}{\it{}unif01\_Gen}\nwendquote}.  It is
{\Tt{}\Rm{}{\it{}unif01\_CreateExtenGenBits}\nwendquote} that allocates the necessary memory for
the generator, but we must call {\Tt{}\Rm{}{\it{}unif01\_DeleteExternGenBits}\nwendquote} when
we're done, so it can free the allocated resources.

\nwenddocs{}\nwbegincode{5}\sublabel{NW1iLVcl-1Pi9cp-1}\nwmargintag{{\nwtagstyle{}\subpageref{NW1iLVcl-1Pi9cp-1}}}\moddef{the main function~{\nwtagstyle{}\subpageref{NW1iLVcl-1Pi9cp-1}}}\endmoddef\Rm{}\nwstartdeflinemarkup\nwusesondefline{\\{NW1iLVcl-1oTVZB-1}}\nwenddeflinemarkup
{\bf{}int} {\it{}main}({\bf{}int} {\it{}argc}, {\bf{}char} \begin{math}\ast\end{math}\begin{math}\ast\end{math}{\it{}argv})
{\nwlbrace}
  {\bf{}char} {\it{}battery}[8]; {\commopen}\begcomm{} values small, medium or big \endcomm{}{\commclose}
  {\bf{}char} {\it{}name}[1000]; {\commopen}\begcomm{} the name of the generator given by user \endcomm{}{\commclose}
  {\bf{}int} {\it{}option}; {\commopen}\begcomm{} the option produced by getopt() \endcomm{}{\commclose}
  {\it{}program\_name} = \begin{math}\ast\end{math}{\it{}argv};
  
  {\it{}memset}({\it{}battery}, {\tt{}'{\char92}0'}, {\bf{}sizeof} {\it{}battery});
  {\it{}memset}({\it{}name}, {\tt{}'{\char92}0'}, {\bf{}sizeof} {\it{}name});

  \LA{}option parsing et cetera~{\nwtagstyle{}\subpageref{NW1iLVcl-3Ncanh-1}}\RA{}
  \LA{}put stdin and stdout in binary mode~{\nwtagstyle{}\subpageref{NW1iLVcl-3NGjJL-1}}\RA{}

  {\it{}unif01\_Gen}\begin{math}\ast\end{math} {\it{}g} = {\it{}unif01\_CreateExternGenBits}({\it{}name}, {\it{}generator});

  {\bf{}if} ({\it{}strncmp}({\tt{}"small"}, {\it{}battery}, {\bf{}sizeof} {\it{}battery}) \begin{math}\equiv\end{math} 0) {\nwlbrace}
    {\it{}bbattery\_SmallCrush}({\it{}g});
  {\nwrbrace} 
  {\bf{}else} 
  {\bf{}if} ({\it{}strncmp}({\tt{}"medium"}, {\it{}battery}, {\bf{}sizeof} {\it{}battery}) \begin{math}\equiv\end{math} 0) {\nwlbrace}
    {\it{}bbattery\_Crush}({\it{}g});
  {\nwrbrace}
  {\bf{}else}
  {\bf{}if} ({\it{}strncmp}({\tt{}"big"}, {\it{}battery}, {\bf{}sizeof} {\it{}battery}) \begin{math}\equiv\end{math} 0) {\nwlbrace}
    {\it{}bbattery\_BigCrush}({\it{}g});
  {\nwrbrace}
  {\bf{}else} {\nwlbrace}
    {\it{}fprintf}({\bf{}stdout}, {\tt{}"never reached{\char92}n"});
  {\nwrbrace}
  
  {\it{}unif01\_DeleteExternGenBits}({\it{}g});
  {\bf{}return} 0;
{\nwrbrace}
\nwused{\\{NW1iLVcl-1oTVZB-1}}\nwendcode{}\nwbegindocs{6}Here's the options we support with sensible user choices checked.

\nwenddocs{}\nwbegincode{7}\sublabel{NW1iLVcl-3Ncanh-1}\nwmargintag{{\nwtagstyle{}\subpageref{NW1iLVcl-3Ncanh-1}}}\moddef{option parsing et cetera~{\nwtagstyle{}\subpageref{NW1iLVcl-3Ncanh-1}}}\endmoddef\Rm{}\nwstartdeflinemarkup\nwusesondefline{\\{NW1iLVcl-1Pi9cp-1}}\nwenddeflinemarkup
  {\bf{}static} {\bf{}struct} {\it{}option} {\it{}long\_options}[] = {\nwlbrace}
    {\nwlbrace}{\tt{}"battery"}, {\it{}required\_argument}, 0,  {\tt{}'b'} {\nwrbrace},
    {\nwlbrace}{\tt{}"name"}, {\it{}required\_argument}, 0,  {\tt{}'n'} {\nwrbrace},
    {\nwlbrace}{\tt{}"help"}, {\it{}no\_argument}, 0, {\tt{}'h'}{\nwrbrace},
    {\nwlbrace}0, 0, 0, 0{\nwrbrace}
  {\nwrbrace};

  {\bf{}int} {\it{}digit\_optind} = 0;
  {\bf{}int} {\it{}option\_index} = 0;
  
  {\bf{}while} (({\it{}option} = {\it{}getopt\_long}({\it{}argc}, {\it{}argv}, {\tt{}"b:n:h"}, {\it{}long\_options}, &{\it{}option\_index})) \begin{math}\neq\end{math} -1)
    {\bf{}switch}({\it{}option}) {\nwlbrace}
      {\bf{}case} {\tt{}'b'}: {\commopen}\begcomm{} b as in battery (of tests) \endcomm{}{\commclose}
      {\bf{}if} ({\it{}is\_valid\_battery}({\it{}optarg}) \begin{math}<\end{math} 0) {\nwlbrace}
          {\it{}fprintf}({\bf{}stdout}, {\tt{}"Error: valid batteries are small, medium, big.{\char92}n"});
          {\it{}usage}();
        {\nwrbrace}
        
        {\commopen}\begcomm{} I assume optarg will never be non-null-terminated. \endcomm{}{\commclose}
        {\it{}strncpy}({\it{}battery}, {\it{}optarg}, {\bf{}sizeof} {\it{}battery});
        {\bf{}break};

      {\bf{}case} {\tt{}'n'}: {\commopen}\begcomm{} n as in name (of the generator) \endcomm{}{\commclose}
        {\it{}strncpy}({\it{}name}, {\it{}optarg}, {\bf{}sizeof} {\it{}name});
        {\bf{}break};
        
      {\bf{}case} {\tt{}'h'}:
      {\bf{}default}:
        {\it{}usage}();
    {\nwrbrace}
  {\it{}argc} -= {\it{}optind};
  {\it{}argv} += {\it{}optind};
    
  {\commopen}\begcomm{} check if user has given all necessary things \endcomm{}{\commclose}
  {\bf{}if} ({\it{}is\_valid\_battery}({\it{}battery}) \begin{math}<\end{math} 0) {\nwlbrace}
    {\it{}fprintf}({\bf{}stdout}, {\tt{}"Error: valid batteries are small, medium, big{\char92}n"});
    {\it{}usage}();
  {\nwrbrace}
  
  {\bf{}if} ({\it{}strlen}({\it{}name}) \begin{math}\equiv\end{math} 0) {\nwlbrace}
    {\it{}fprintf}({\bf{}stdout}, {\tt{}"Error: you must give a name to your generator{\char92}n"});
    {\it{}usage}();
  {\nwrbrace}
\nwused{\\{NW1iLVcl-1Pi9cp-1}}\nwendcode{}\nwbegindocs{8}On Win32 systems, we must explicitly put {\Tt{}\Rm{}{\bf{}stdin}\nwendquote} and {\Tt{}\Rm{}{\bf{}stdout}\nwendquote} in
binary mode.  Otherwise, there'll be some byte translations occuring
on our backs.  For example, if you write byte {\Tt{}\Rm{}{\tt{}'{\char92}n'}\nwendquote} to the
{\Tt{}\Rm{}{\bf{}stdout}\nwendquote}, the system will make sure to add byte {\Tt{}\Rm{}{\tt{}'{\char92}r'}\nwendquote} before
writing byte {\Tt{}\Rm{}{\tt{}'{\char92}n'}\nwendquote} because Win32 systems use the sequence
{\Tt{}\Rm{}{\tt{}"{\char92}r{\char92}n"}\nwendquote} as line separator.  There seems to be no way of doing this
portably.

\nwenddocs{}\nwbegincode{9}\sublabel{NW1iLVcl-3NGjJL-1}\nwmargintag{{\nwtagstyle{}\subpageref{NW1iLVcl-3NGjJL-1}}}\moddef{put stdin and stdout in binary mode~{\nwtagstyle{}\subpageref{NW1iLVcl-3NGjJL-1}}}\endmoddef\Rm{}\nwstartdeflinemarkup\nwusesondefline{\\{NW1iLVcl-1Pi9cp-1}}\nwenddeflinemarkup
{\bf{}\char35{}ifdef}{\tt{} WIN32}
  {\it{}\_setmode}({\it{}\_fileno}( {\bf{}stdin}), {\it{}\_O\_BINARY});
  {\it{}\_setmode}({\it{}\_fileno}({\bf{}stdout}), {\it{}\_O\_BINARY});
{\bf{}\char35{}endif}{\tt{}}
\nwused{\\{NW1iLVcl-1Pi9cp-1}}\nwendcode{}\nwbegindocs{10}\section*{Other chunks}

Nothing out of the ordinary here.

\nwenddocs{}\nwbegincode{11}\sublabel{NW1iLVcl-3EEkjx-1}\nwmargintag{{\nwtagstyle{}\subpageref{NW1iLVcl-3EEkjx-1}}}\moddef{other functions~{\nwtagstyle{}\subpageref{NW1iLVcl-3EEkjx-1}}}\endmoddef\Rm{}\nwstartdeflinemarkup\nwusesondefline{\\{NW1iLVcl-1oTVZB-1}}\nwenddeflinemarkup
{\bf{}int} {\it{}is\_valid\_battery}({\bf{}char} \begin{math}\ast\end{math}{\it{}optarg}) {\nwlbrace}
  {\bf{}if} ({\it{}strncmp}({\tt{}"small"}, {\it{}optarg}, {\bf{}sizeof} {\tt{}"small"}) \begin{math}\equiv\end{math} 0) {\bf{}return} 0;
  {\bf{}if} ({\it{}strncmp}({\tt{}"medium"}, {\it{}optarg}, {\bf{}sizeof} {\tt{}"medium"}) \begin{math}\equiv\end{math} 0) {\bf{}return} 0;
  {\bf{}if} ({\it{}strncmp}({\tt{}"big"}, {\it{}optarg}, {\bf{}sizeof} {\tt{}"big"}) \begin{math}\equiv\end{math} 0) {\bf{}return} 0;
  {\bf{}return} -1;
{\nwrbrace}

{\bf{}void} {\it{}usage}({\bf{}void}) {\nwlbrace}
  {\it{}fprintf}({\bf{}stdout}, {\tt{}"Usage: %s [options]{\char92}n"}, {\it{}program\_name});
  {\it{}fprintf}({\bf{}stdout}, {\tt{}" Tests your data for randomness against TestU01.{\char92}n{\char92}n"});
  {\it{}fprintf}({\bf{}stdout}, {\tt{}"Examples:{\char92}n"});
  {\it{}fprintf}({\bf{}stdout}, {\tt{}"cat /dev/urandom | crush -b small -n 'the local generator'{\char92}n"});
  {\it{}fprintf}({\bf{}stdout}, {\tt{}"xorshift32 | crush -b small -n xorshift32{\char92}n{\char92}n"});
  {\it{}fprintf}({\bf{}stdout}, {\tt{}" The options are:{\char92}n"}
  {\tt{}"-b, --battery    Your choice of battery (small, medium, big){\char92}n"}
  {\tt{}"-n, --name       The name of your generator{\char92}n"}
  {\tt{}"-h, --help       Display this information{\char92}n"});
  {\bf{}exit}(0);
{\nwrbrace} 
\nwindexdefn{\nwixident{is{\_}valid{\_}battery}}{is:unvalid:unbattery}{NW1iLVcl-3EEkjx-1}\nwindexdefn{\nwixident{usage}}{usage}{NW1iLVcl-3EEkjx-1}\eatline
\nwused{\\{NW1iLVcl-1oTVZB-1}}\nwidentdefs{\\{{\nwixident{is{\_}valid{\_}battery}}{is:unvalid:unbattery}}\\{{\nwixident{usage}}{usage}}}\nwendcode{}\nwbegindocs{12}\nwdocspar
\nwenddocs{}\nwbegincode{13}\sublabel{NW1iLVcl-2UDuHk-1}\nwmargintag{{\nwtagstyle{}\subpageref{NW1iLVcl-2UDuHk-1}}}\moddef{header and declarations~{\nwtagstyle{}\subpageref{NW1iLVcl-2UDuHk-1}}}\endmoddef\Rm{}\nwstartdeflinemarkup\nwusesondefline{\\{NW1iLVcl-1oTVZB-1}}\nwenddeflinemarkup
{\bf{}\char35{}include}{\tt{} \begin{math}<\end{math}stdlib.h\begin{math}>\end{math}}
{\bf{}\char35{}include}{\tt{} \begin{math}<\end{math}stdio.h\begin{math}>\end{math}}
{\bf{}\char35{}include}{\tt{} \begin{math}<\end{math}stdint.h\begin{math}>\end{math}}
{\bf{}\char35{}include}{\tt{} \begin{math}<\end{math}string.h\begin{math}>\end{math}}
{\bf{}\char35{}include}{\tt{} \begin{math}<\end{math}unistd.h\begin{math}>\end{math}}
{\bf{}\char35{}include}{\tt{} \begin{math}<\end{math}getopt.h\begin{math}>\end{math}}
{\bf{}\char35{}ifdef}{\tt{} WIN32}
{\bf{}\char35{}include}{\tt{} \begin{math}<\end{math}fcntl.h\begin{math}>\end{math} /* \_O\_BINARY */}
{\bf{}\char35{}include}{\tt{} \begin{math}<\end{math}io.h\begin{math}>\end{math}  /* \_setmode() */}
{\bf{}\char35{}endif}{\tt{}}
{\bf{}\char35{}include}{\tt{} "TestU01.h"}

{\bf{}void} {\it{}usage}({\bf{}void});
{\bf{}int} {\it{}is\_valid\_battery}({\bf{}char} \begin{math}\ast\end{math}{\it{}optarg});
{\bf{}int} {\it{}min}({\bf{}int} {\it{}a}, {\bf{}int} {\it{}b});
{\bf{}char} \begin{math}\ast\end{math}{\it{}program\_name}; {\commopen}\begcomm{} a pointer to argv[0] \endcomm{}{\commclose}
\nwused{\\{NW1iLVcl-1oTVZB-1}}\nwendcode{}

\nwixlogsorted{c}{{a generic generator}{NW1iLVcl-PfqK8-1}{\nwixu{NW1iLVcl-1oTVZB-1}\nwixd{NW1iLVcl-PfqK8-1}}}%
\nwixlogsorted{c}{{crush.c}{NW1iLVcl-1oTVZB-1}{\nwixd{NW1iLVcl-1oTVZB-1}}}%
\nwixlogsorted{c}{{header and declarations}{NW1iLVcl-2UDuHk-1}{\nwixu{NW1iLVcl-1oTVZB-1}\nwixd{NW1iLVcl-2UDuHk-1}}}%
\nwixlogsorted{c}{{option parsing et cetera}{NW1iLVcl-3Ncanh-1}{\nwixu{NW1iLVcl-1Pi9cp-1}\nwixd{NW1iLVcl-3Ncanh-1}}}%
\nwixlogsorted{c}{{other functions}{NW1iLVcl-3EEkjx-1}{\nwixu{NW1iLVcl-1oTVZB-1}\nwixd{NW1iLVcl-3EEkjx-1}}}%
\nwixlogsorted{c}{{put stdin and stdout in binary mode}{NW1iLVcl-3NGjJL-1}{\nwixu{NW1iLVcl-1Pi9cp-1}\nwixd{NW1iLVcl-3NGjJL-1}}}%
\nwixlogsorted{c}{{the main function}{NW1iLVcl-1Pi9cp-1}{\nwixu{NW1iLVcl-1oTVZB-1}\nwixd{NW1iLVcl-1Pi9cp-1}}}%
\nwixlogsorted{i}{{\nwixident{generator}}{generator}}%
\nwixlogsorted{i}{{\nwixident{is{\_}valid{\_}battery}}{is:unvalid:unbattery}}%
\nwixlogsorted{i}{{\nwixident{usage}}{usage}}%
\nwbegindocs{14}\nwdocspar
\printbibliography

\section*{Chunk list}
\nowebchunks

\section*{Index}
\nowebindex
 
\end{document}
\nwenddocs{}
